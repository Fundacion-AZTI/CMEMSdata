% Options for packages loaded elsewhere
\PassOptionsToPackage{unicode}{hyperref}
\PassOptionsToPackage{hyphens}{url}
%
\documentclass[
]{book}
\usepackage{amsmath,amssymb}
\usepackage{iftex}
\ifPDFTeX
  \usepackage[T1]{fontenc}
  \usepackage[utf8]{inputenc}
  \usepackage{textcomp} % provide euro and other symbols
\else % if luatex or xetex
  \usepackage{unicode-math} % this also loads fontspec
  \defaultfontfeatures{Scale=MatchLowercase}
  \defaultfontfeatures[\rmfamily]{Ligatures=TeX,Scale=1}
\fi
\usepackage{lmodern}
\ifPDFTeX\else
  % xetex/luatex font selection
\fi
% Use upquote if available, for straight quotes in verbatim environments
\IfFileExists{upquote.sty}{\usepackage{upquote}}{}
\IfFileExists{microtype.sty}{% use microtype if available
  \usepackage[]{microtype}
  \UseMicrotypeSet[protrusion]{basicmath} % disable protrusion for tt fonts
}{}
\makeatletter
\@ifundefined{KOMAClassName}{% if non-KOMA class
  \IfFileExists{parskip.sty}{%
    \usepackage{parskip}
  }{% else
    \setlength{\parindent}{0pt}
    \setlength{\parskip}{6pt plus 2pt minus 1pt}}
}{% if KOMA class
  \KOMAoptions{parskip=half}}
\makeatother
\usepackage{xcolor}
\usepackage{longtable,booktabs,array}
\usepackage{calc} % for calculating minipage widths
% Correct order of tables after \paragraph or \subparagraph
\usepackage{etoolbox}
\makeatletter
\patchcmd\longtable{\par}{\if@noskipsec\mbox{}\fi\par}{}{}
\makeatother
% Allow footnotes in longtable head/foot
\IfFileExists{footnotehyper.sty}{\usepackage{footnotehyper}}{\usepackage{footnote}}
\makesavenoteenv{longtable}
\usepackage{graphicx}
\makeatletter
\def\maxwidth{\ifdim\Gin@nat@width>\linewidth\linewidth\else\Gin@nat@width\fi}
\def\maxheight{\ifdim\Gin@nat@height>\textheight\textheight\else\Gin@nat@height\fi}
\makeatother
% Scale images if necessary, so that they will not overflow the page
% margins by default, and it is still possible to overwrite the defaults
% using explicit options in \includegraphics[width, height, ...]{}
\setkeys{Gin}{width=\maxwidth,height=\maxheight,keepaspectratio}
% Set default figure placement to htbp
\makeatletter
\def\fps@figure{htbp}
\makeatother
\setlength{\emergencystretch}{3em} % prevent overfull lines
\providecommand{\tightlist}{%
  \setlength{\itemsep}{0pt}\setlength{\parskip}{0pt}}
\setcounter{secnumdepth}{5}
\usepackage{booktabs}
\usepackage{listings}
\lstset{breaklines=true}
\ifLuaTeX
  \usepackage{selnolig}  % disable illegal ligatures
\fi
\usepackage[]{natbib}
\bibliographystyle{plainnat}
\usepackage{bookmark}
\IfFileExists{xurl.sty}{\usepackage{xurl}}{} % add URL line breaks if available
\urlstyle{same}
\hypersetup{
  pdftitle={Download and prepare data from the Copernicus Marine Environment Monitoring Service (CMEMS)},
  pdfauthor={AZTI},
  hidelinks,
  pdfcreator={LaTeX via pandoc}}

\title{Download and prepare data from the Copernicus Marine Environment Monitoring Service (CMEMS)}
\author{AZTI}
\date{2025-03-03}

\begin{document}
\maketitle

{
\setcounter{tocdepth}{1}
\tableofcontents
}
\chapter*{About}\label{about}
\addcontentsline{toc}{chapter}{About}

This is a short tutorial explaining how to download and prepare data from the Copernicus Marine Environment Monitoring Service (CMEMS).

The code is available in \href{https://github.com/Fundacion-AZTI/gam-niche}{AZTI's github repository} and the book is readily available \href{https://fundacion-azti.github.io/gam-niche/}{here}.This work is licensed under a \href{https://creativecommons.org/licenses/by-nc-sa/4.0/}{Creative Commons Attribution-NonCommercial-ShareAlike 4.0 International License (CC BY-NC-SA 4.0)}

To cite this book, please use:

Valle, M., Citores, L., Ibaibarriaga, L., Chust, C. (2023) GAM-NICHE: Shape-Constrained GAMs to build Species Distribution Models under the ecological niche theory. AZTI. \url{https://doi.org/10.57762/fzpy-6w51}

\chapter{Introduction}\label{introduction}

Species Distribution Models (SDMs) are numerical tools that combine observations of species occurrence or abundance at known locations with information on the environmental and/or spatial characteristics of those locations \citep{elith_etal_2009}. SDMs are widely used as a tool for understanding species spatial ecology and are also known as ecological niche models (ENM) or habitat suitability models.

According to ecological niche theory, species response curves are unimodal with respect to environmental gradients \citep{hutchinson_1957}. While a variety of statistical methods have been developed for species distribution modelling, a general problem with most of these habitat modelling approaches is that the estimated response curves can display biologically implausible shapes which do not respect ecological niche theory. This is because species response curves are fit statistically with any assumption or restriction, which sometimes do not respect the ecological niche theory. To better understand species response to environmental changes, SDMs should consider theoretical background such as the ecological niche theory and pursue the unimodality of the response curves with respect to environmental gradients.

This book provides a tutorial on how to use Shape-Constrained Generalized Additive Models (SC-GAMs) \citep{pya_etal_2015} to build SDMs under the ecological niche theory framework \citep{citores_etal_2020}. SC-GAMs impose monotonicity and concavity constraints in the linear predictor of the GAMs and avoid overfitting. SC-GAM is an effective alternative to fitting nonsymmetric parametric response curves, while retaining the unimodality constraint, required by ecological niche theory, for direct variables and limiting factors.

The book is organised following the key steps in good modelling practice of SDMs \citep{elith_etal_2009}. First, presence data of a selected species are downloaded from GBIF/OBIS global public datasets and pseudo-absence data are created. Then, environmental data are downloaded from public repositories and extracted at each of the presence/pseudo-absence data points. Based on this dataset, an exploratory analysis is conducted to help deciding on the best modelling approach. The model is fitted to the dataset and the quality of the fit and the realism of the fitted response function are evaluated. After selecting a threshold to transform the continuous probability predictions into binary responses, the model is validated using a k-fold approach. Finally, the predicted maps are generated for visualization.

  \bibliography{references.bib}

\end{document}
